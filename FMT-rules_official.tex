\documentclass[12pt,a4paper]{article}
\usepackage[utf8]{inputenc}
\usepackage[english,russian]{babel}
\usepackage{amsmath}
\usepackage{amssymb,nccmath}
\usepackage{geometry,indentfirst,color}
\usepackage[pdftex]{graphicx}
\usepackage{sidecap}
\geometry{top=0.5cm} %поле сверху
\geometry{bottom=0.5cm} %поле снизу
\geometry{left=1.5cm} %поле справа
\geometry{right=1.5cm}
\usepackage{wrapfig}
\usepackage{epigraph}
%\pagestyle{empty}
\DeclareGraphicsExtensions{.pdf,.png,.jpg}
\DeclareMathOperator{\Tr}{Tr}
\setlength{\textheight}{26cm}\newcounter{nnn}\setlength{\topmargin}{-20mm}\newcounter{qqq}
\begin{document}
\begin{flushleft}
\parbox[t][0pt]{0.2\textwidth}
{
{%\centering
\vspace{0\baselineskip}
\includegraphics[scale=1.5]{klsh_logo_mod.pdf}\par
}
}
\end{flushleft}
\hfill
\parbox[t][0pt]{0.80\textwidth}
{
{\centering
\vspace{-1.5\baselineskip}
\begin{center}
{\Huge Правила ФМТ}\\
{Красноярская Летняя Школа $2022$}\par
\end{center}
}
}
\vspace{2.1\baselineskip}

\section{Общие положения}
Физико-математический турнир (ФМТ)~--- командное соревнование по решению задач по физике и математике. Ниже используются следующие понятия:
\begin{list}{\arabic{qqq}.}{\usecounter{qqq}\leftmargin=10mm \labelwidth=5mm \topsep=0mm \labelsep=2mm \itemsep=1pt \parsep=0mm \itemindent=-15pt}
\item \textbf{Команда КЛШ} --- совокупность школьников, составляющих одну из 24 команд КЛШ, идентифицируемых греческими буквами;
\item \textbf{Команда ФМТ} --- не более 4 представителей команды КЛШ, которые участвуют в соревновании;
\item \textbf{Тур} --- один из раундов соревнований, в ходе которого встречаются все 24 команды, распределённые на пары. В ФМТ летнего сезона 2019 года будет проведено 6 туров;
\item \textbf{Основной этап} --- первая часть любого тура, во время которого обе команды ФМТ решают одни и те же задачи, предлагаемые судьями. В ФМТ летнего сезона 2019 года основной этап включает в себя 2 задачи по физике и 2 --- по математике и продолжается не менее 20 минут;
\item \textbf{Этап обмена ударами} --- вторая часть любого тура, во время которого обе команды решают \textbf{обменные задачи};
\item \textbf{Обменная задача} --- задача, которую каждая команда ФМТ обязана приготовить заранее для того, чтобы предложить противоположной команде, принимающей участие в \textbf{туре} для решения. В сезоне КЛШ 2019 года обменные задачи могут быть по физике и математике, при этом обменные задачи не могут иметь решение, основывающееся на материале, выходящем за пределы программы 8 -- 9 классов общеобразовательных школ;
\item \textbf{Жеребьёвка} --- специально организованная процедура, в ходе которой формируются пары для проведения первого \textbf{тура};
\item \textbf{Свалка} --- специальный (седьмой) этап, проводимый по отдельной процедуре;
\item \textbf{Финал и супер-финал} --- окончательные туры, проводимые по регулярным правилам, в которых встречаются 4 команды, имеющие самый высокий рейтинг по результатам 6 \textbf{основных туров} и \textbf{свалки};
\item \textbf{Докладчик} --- член команды ФМТ, излагающий решение заявленной командой задачи у доски;
\item \textbf{Оппонент} --- член другой команды ФМТ, слушающий и анализирующий решение, излагаемое \textbf{докладчиком};
\item \textbf{Вольные стрелки} --- члены команды КЛШ, не являющиеся членами команды ФМТ и принимающие участие в туре по специальным правилам;
\end{list}

Основная часть турнира~--- 6 туров~--- пройдут в течение сезона. Встреча команд в первом туре определяется жеребьёвкой. В последний день сезона будет проведена свалка. После неё в этот же день проводятся финал и суперфинал. Команды, занявшие первое, второе и третье места, награждаются ценными призами.

\section{Жеребьёвка}
Жеребьёвка проводится для всех команд ФМТ одновременно и представляет собой соревнование по решению задач по физике и математике на скорость; место проведения жеребьёвки указывается в распорядке дня заранее. В жеребьёвке участвует вся команда КЛШ. Каждая команда КЛШ работает со своим судьёй. В начале жеребьёвки команда КЛШ получает четыре лёгкие задачи: две по физике и две по математике, правильное решение каждой из которых оценивается в один балл. На то, чтобы сдать одну задачу, у команды есть три попытки. \textbf{Сдать} задачу означает предъявить судьям единственный ответ.

За правильное решение любой из 2 задач по математике (по физике, соответственно) на 1 балл команда с первой попытки получает 1 балл за задачу, за правильное решение со второй попытки --- 0,5 балла, за правильное решение с третьей попытки --- 0 баллов. После сдачи задач на 1 балл команда получает у судьи 1 задачу по физике и 1 по математике, стоимостью по 2 балла каждая. 
На сдачу этих задач у команды также есть три попытки, а баллы за решение начисляются аналогично: 2 балла за сдачу с первой попытки, 1 балл за сдачу со второй и 0 баллов --- за сдачу с третьей попытки. 
После сдачи задач на 2 балла команда получает задачи, за решение каждой из которых можно набрать до 4 баллов. 
Способ начисления баллов в зависимости от номера попытки полностью аналогичен описанному.

Жеребьёвка длится 15 минут. Результаты жеребьёвки учитываются при определении общего рейтинга команд с весом $\mfrac{1}{2}$.

\section{Основной тур}
Основной тур состоит из двух этапов: \textsl{основной} этап и \textsl{обмен ударами}. К началу тура команда ФМТ должна явиться вовремя, имея при себе {\bf два} идентичных экземпляра обменной задачи, напечатанных или написанных разборчивым почерком на бумажном носителе.

\subsection{Основной этап}
Основной этап длится 20 минут. В начале этапа команды ФМТ получают одинаковые комплекты четырёх задач: две по физике и две по математике. Цель команды ФМТ --- решить все задачи основного тура раньше команды ФМТ соперника. Далее для простоты обозначений будем считать, что в туре участвуют команды $\alpha$ и $\beta$.

Если команда считает, что решила задачу, то один из членов команды поднимает руку и громко произносит: \textsl{Команда \texttt{название} заявляет задачу номер \texttt{№ задачи}}. В этот момент судьи в обязательном порядке опрашивают вольных стрелков и принимают от них решение заявленной задачи в письменном виде. Затем судья сразу же просит представителя команды-заявительницы сообщить ответ и фиксирует его на доске, а также приглашает к доске докладчика и оппонента.

Пусть задачу заявила команда ФМТ $\alpha$. Докладчик от команды $\alpha$ выходит к доске для доклада, а команда $\beta$ обязана выставить оппонента. По приглашению судьи \textbf{докладчик} излагает решение; он может пользоваться собственными записями, сделанными во время работы над заявленной задачей за столом. Докладчик обязан сделать доклад не прерываясь на долгие паузы; время доклада у доски регламентируется судьями встречи.

После окончания \textbf{доклада} представитель противоположной команды ФМТ получает возможность для \textbf{оппонирования}. Для этого судья спрашивает оппонента, всё ли ему понятно в изложенном решении, и если оппоненту что-то осталось непонятным, то оппонент может задать вопрос(ы) для понимания. Далее оппонент может высказать свои замечания по решению, предложенному командой-заявительницей.

После доклада и оппонирования судьи принимают решение о судьбе задачи. Здесь возможны три варианта:
\begin{list}{--}{\leftmargin=10mm \labelwidth=5mm \topsep=0mm \labelsep=2mm \itemsep=1pt \parsep=0mm \itemindent=-1.5pt}
\item задача решена верно командой-заявительницей, ответ правильный, команда-заявительница получает за решение 2 балла, задача снимается;
\item задача решена командой-заявительницей неверно, ответ неправильный, переход подачи;
\item задача решена неверно, однако ответ у команды-заявительницы правильный. Задача снимается, команда-заявительница получает один балл за правильный ответ.
\end{list}
Судьи также принимают решение по оппонированию. 
Если судьи находят, что оппонент нашёл существенный пробел в решении задачи, они могут дать один балл оппонирующей команде.

В случае \textbf{перехода подачи} команды продолжают решение задачи, однако безусловное право заявки переходит теперь к команде оппонента ($\beta$ в нашем примере) и далее судьба задачи определяется регулярными правилами. 
Число переходов подачи по каждой задаче ограничено лишь временем проведения основного тура. 
Решение каждой задачи принимается и оценивается независимо (и, может быть, параллельно во времени) друг от друга.

{\bf Вольные стрелки}: каждая КЛШ-команда вправе выставить на встречу любое количество вольных стрелков из своего состава. 
Вольные стрелки получают те же задачи основного тура, что и их команда ФМТ. 
Если вольные стрелки считают, что они решили какую-то задачу, то они должны оформить решение задачи в письменном виде и передать его судейской бригаде \textbf{до} того, как ответ на эту задачу будет объявлен. За каждую задачу вольные стрелки могут заработать до двух очков. Стрелки могут заявлять любую из четырёх задач, но в любом случае они смогут принести своей команде ФМТ \textbf{не более двух} очков. Если задача не снята судьями после изложения решения и оппонирования, вольные стрелки могут продолжить решать эту задачу.

\subsection{Обмен ударами}
Следом за основным этапом наступает этап обмена ударами; его продолжительность составляет 10 минут. Командам даётся 10 минут на решение задачи соперника. Первой докладывать задачу оппонента будет та команда, которая по итогам основного этапа набрала больше баллов. Обменные задачи докладываются по истечении времени, отведённого на этап.

После того, как обе команды завершили докладывать решения обменных задач, обе команды-хозяйки обменных задач должны продемонстрировать \textbf{знание решения своей задачи}. Решение своей задачи нужно доложить \textbf{за три минуты}. Собственное решение нужно докладывать полностью по памяти, все вычисления проделывать на доске в процессе доклада. Решение может докладывать {\bf любой} член команды-хозяйки по желанию самой команды.

Если команда-хозяйка не уложилась в предоставленное время или решение оказалось неверным, то эта команда получит штраф в один или два балла.

\section{Штрафы}
Опоздание, громкие разговоры и невежливое поведение, мешающее судьям и противоположной команде ФМТ, наказываются дисциплинарным штрафом до двух баллов. 
Незнание решения собственной обменной задачи наказывается штрафом до двух баллов.

\section{Подписание протоколов и апелляция}
По завершении тура команды ФМТ ставят свои подписи в протоколе встречи. Если команда не согласна с {\bf действиями судей} на одном из этапов (или в обоих), то вместо подписи капитан команды должен написать слово ``\textsl{апелляция}''. Затем в течение {\bf одного часа} апеллирующая команда ФМТ должна предоставить главному судье ФМТ изложение своих претензий в письменном виде. Для рассмотрения этих претензий главным судьёй ФМТ создаётся апелляционная комиссия; её решения являются окончательными. Если комиссия решит, что претензии обоснованы, то счёт той части тура (основного времени или обмена ударами), которая была оспорена, становится равным $0:0$.

\section{Между основными турами}
В течение всей школы мы ведём турнирную таблицу, стараясь делать так, чтобы между собой играли близкие по силе команды, при этом не допускаем, чтобы две команды встречались дважды. Положение в таблице определяется суммой очков, набранных командой за прошедшие туры: за победу команда получает два очка турнирной таблицы, за ничью --- одно очко\footnote{Кроме того, мы учитываем баллы, набранные в туре за решение задач, чтобы снять вырождение при сортировке команд.}. По итогам всех основных туров каждая команда получает от 0 до 23 {\bf финальных очков}, равное числу команд, сыгравших хуже данной.

\section{Разбор задач}
После каждого тура судейская бригада организует разбор задач основного этапа.

\section{Свалка}
Свалка проходит в последний день КЛШ и в ней участвуют все команды ФМТ одновременно; 
место проведения свалки объявляется заранее в распорядке дня. 
Каждая команда борется не против одного соперника, а против {\bf хронометра}.   
Каждая команда работает со своим судьей. По сигналу судьи-хронометриста все команды ФМТ получают 4 задачи. 
Каждая команда решает задачи так, чтобы судья видел ход решения, и заявляет свой ответ на специальном бланке. 
Судья фиксирует время заявки и баллы за решение: ноль, один или два, в зависимости от правильности ответа и решения. 
Заявлять задачи на свалке можно в произвольном порядке, на каждую задачу отводится только \textbf{одна} заявка.

\subsection{Рейтинг в свалке}

\begin{enumerate}
	\item За каждую задачу команда может получить до 1200 штрафных секунд. 
	Число штрафных секунд за задачу определяется следующим образом: если задача решена на 0 баллов, команда получает 1200 штрафных секунд; если задача решена на 1 балл, команда получает штрафных секунд столько, сколько секунд прошло с момента начала свалки до момента заявки задачи; если задача решена на 2 балла, команда получает {\bf половину} от числа секунд, прошедших с момента начала свалки до момента заявки задачи;
	\item Штрафные секунды по задачам суммируются, таким образом команда может набрать до 4800 штрафных секунд;
	\item Мы сортируем команды по возрастанию числа штрафных секунд, и каждая команда получает столько {\bf финальных очков}, сколько команд выступило хуже неё (от 0 до 23).
\end{enumerate}


\section{Финал и суперфинал}
Суперфинал и финал проводятся по правилам обычного тура за следующими исключениями:
\begin{enumerate}
\item нет вольных стрелков;
\item судейская бригада {\bf самостоятельно} определяет, кто из членов команды ФМТ будет докладывать решение своей задачи на обмене ударами.

\end{enumerate}
\end{document}
