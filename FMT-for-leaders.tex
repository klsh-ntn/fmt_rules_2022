\documentclass[12pt]{article}
\usepackage[utf8]{inputenc}
\usepackage[english,russian]{babel}
\usepackage{amsmath}
\usepackage{amssymb}
\usepackage{geometry,indentfirst,color}
\usepackage[pdftex]{graphicx}
\usepackage{sidecap}
\geometry{top=0.5cm} %поле сверху
\geometry{bottom=0.5cm} %поле снизу
\geometry{left=1.5cm} %поле справа
\geometry{right=1.5cm}
\usepackage{wrapfig}
\usepackage{epigraph}
%\pagestyle{empty}
\DeclareGraphicsExtensions{.pdf,.png,.jpg}
\DeclareMathOperator{\Tr}{Tr}
\setlength{\textheight}{24cm}\newcounter{nnn}\setlength{\topmargin}{-20mm}
\begin{document}
\begin{flushleft}
\parbox[t][0pt]{0.2\textwidth}
{
{%\centering
\vspace{0\baselineskip}
\includegraphics[scale=1.5]{klsh_logo_mod.pdf}\par
}
}
\end{flushleft}
\hfill
\parbox[t][0pt]{0.80\textwidth}
{
{\centering
\vspace{-1.5\baselineskip}
\begin{flushright}
{\Huge ФМТ}\\
{Красноярская Летняя Школа $2022$}
\end{flushright}
}
}
\vspace{4\baselineskip}

\section*{<<Памятка для вожатых>> или <<Что нужно делать, чтобы команда была счастлива?>>}
Пока команда воюет на турнире, Вы можете сделать очень многое для её успеха. В первую очередь, убедитесь, что ребята достаточно внимательно прочли правила турнира до <<тест-драйва>>. Кроме того...
\subsection*{Обмен ударами}
В течение сезона ожидается 6 туров ФМТ, из них один игровой. На пять остальных туров (и на финал, если вашей команде повезёт) команда должна {\bf подготовить задачу} на обмен ударами. Какая задача считается хорошей, хорошо описано в печатной версии правил ФМТ. Вкратце:

\begin{enumerate}
	\item задача должна быть интересной и элегантной (по возможности);
	\item решение задачи можно не торопясь полностью рассказать у доски за {\bf три} минуты, иначе на команда получает большой штраф;
	\item команда должна подготовить к началу тура два одинаковых листочка с условием задачи: один для судей, другой для соперников;
	\item если Вы не можете оценить задачу, которую ваша команда предлагает на турнир, (настойчиво) предложите сходить к сотруднику НТН за советом. Нашим сотрудникам можно доверять, <<слива>> не будет, но команда с гораздо меньшей вероятностью получит штраф;
\end{enumerate}

\subsection*{Вольные стрелки}
Кроме четырёх человек, непосредственно играющих в турнир, Ваша команда может послать всех остальных участников в так называемых вольных стрелков. Вольные стрелки в течение турнира тоже могут решать задачи, но оформляют решение письменно. 
\begin{enumerate}
	\item Вольные стрелки не находятся под формальным прессом правил и чувствуют себя гораздо более расслабленно. Иногда даже сильные игроки уходят в вольные стрелки, чтобы спокойно решать (но это редкость).
	\item Наличие вольных стрелков поднимает моральный дух команды в поле.
	\item Вольные стрелки реально могут изменить ход матча, так как могут принести до двух очков своей команде, а это, на самом деле, много.
\end{enumerate}

Часто школьники не заинтересованы в том, чтобы быть вольными стрелками (<<мы ничего не решим>> и\,т.\,д.). Но для Вас это~--- отличный способ сплотить команду, ибо победу (или поражение) тогда они будут делить вместе.

Если команда говорит, что они ничего не решат как вольные стрелки, напомните им, что (честное слово!) мы каждый тур даём одну очень простую задачу, которую решит любой школьник, даже не с НТН (дело лишь в скорости). 

\subsection*{Мотивация}
Вам так или иначе стоит подогревать мотивацию ребят к хорошему выступлению на турнире. 
\begin{enumerate}
	\item Если команда много проигрывает, уделите больше времени подготовке обменных задач.
	\item Помните про <<свалку>>. Этот тур, который проходит в последний день школы, имеет вес, равный всем остальным турам, вместе взятым! Свалка может команду с десятого места вывести в финал. 
	\item Если Вы стоите на женской команде, внимательно отнеситесь к высказываниям вроде <<завтра играем против мальчиков, нам кирдык>>. У женских команд неплохой win-record в КЛШ. Победа в 2008, второе место в 2010, победа в 2013, суперфинал в 2017, победа в 2018 и ещё много примеров. Их было бы ещё больше, если бы девочки {\bf не боялись} играть с мальчиками. В ФМТ кроме умения решать задачи очень важно быть смелым и быстро заявлять, но часто девочки из-за страха этого просто не делают. Объясните им, что всё в их руках, и что им надо быть смелыми тоже.
	\item Обязательно приходите время от времени поболеть за свою команду сами! Это поможет вам скорее сблизиться с командой и позволит в более тонких деталях понять их эмоции от прошедшего турнира. Кроме того, Вы можете помочь следить за поведением :)
	\item Если ничего из вышеупомянутого не сработало, обратитесь к Никите Астраханцеву или Борису Демешеву, они придут к Вам в команду и подбодрят её (это не шутка).
\end{enumerate}

\subsection*{Конфликт с судейской бригадой}
Штраф за незнание собственной задачи обычно ложится тяжким грузом на настроение школьников. Старший судья всегда мотивирует своё решение, однако школьники могут не понять его или (вполне разумно) не согласиться с ним. 

Судейские бригады иногда допускают ошибки, и, если команда уверена, что их оштрафовали неправомерно (или что оппонента должны были оштрафовать, но не оштрафовали), или что была допущена ошибка в первом туре, команда может подать апелляцию, которую мы непредвзято рассмотрим тем же вечером. Объясните команде, что нет ничего плохого в том, чтобы подать апелляцию, но не нужно подавать её по любому поводу. Апелляция --- это не выбивание баллов из судей, а оспаривание реальных недочётов. 

Если команда не согласна с решением судей, но по какой-то причине не подала апелляцию, попросите подойти недовольных школьников к старшему судье ещё раз, чтобы тот мотивировал свой поступок.

\subsection*{Состав команды и ротация}
По правилам мы не дисквалифицируем команду, если в поле выходят не четыре, а три или, например, два участника: всякое бывает. Но Вы должны сделать всё возможное, чтобы этого не происходило: найти замену, уговорить кого-то. Кроме того, состав команды от тура к туру может меняться. Если у Вас в команде более четырёх человек хочет участвовать в ФМТ, попробуйте сделать так, чтобы они все попробовали поиграть, пусть никто не будет обиженным. Школьник, временно не играющий в команде, может заработать очков как вольный стрелок.
\end{document} 
